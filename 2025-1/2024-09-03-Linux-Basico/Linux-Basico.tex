%%%%%%%%%%%%%%%%%%%%%%%%%%%%%%%%%%%%%%%%%%%%%%%%%%%%%%%%%%%%%%%%%%%%%%%%%%%%%%%%%%%%%%%%%
% Autor:        Aguilar Enriquez, Paul Sebastian a.k.a. Penserbjorne
% Autor:        @yens7
% Autor:        Calderón Olalde, Enrique Job @ksobrenat32
% Fecha:        05/01/2024
% Descripción:  Plantilla base para cursos.
%%%%%%%%%%%%%%%%%%%%%%%%%%%%%%%%%%%%%%%%%%%%%%%%%%%%%%%%%%%%%%%%%%%%%%%%%%%%%%%%%%%%%%%%%

\documentclass[a4paper,11pt]{article}                 % Papel tamaño carta, texto de 11pt.

\usepackage[top=2cm, bottom=2cm, left=2.2cm, right=2.2cm]{geometry} % Margenes
\usepackage[T1]{fontenc}                              % Indicamos la codificación de las fuentes.
\usepackage[utf8x]{inputenc}                          % Definimos la codificación.
\usepackage{lmodern}                                  % Para poder usar acentos.
\usepackage[spanish]{babel}                           % Usaremos idioma español.
\usepackage{amsmath}                                  % Para formulas matemáticas.
\usepackage{graphicx}                                 % Para imágenes.
\usepackage{float}                                    % Para posicionar objetos.
\usepackage{booktabs}                                 % Para formatear tablas.
\usepackage{hyperref}                                 % Para enlaces y referencias.

\usepackage[table]{xcolor}


%%%%%%%%%%%%%%%%%%%%%%%%%%%%%%%%%%%%%%%%%%%%%%%%%%%%%%%%%%%%%%%%%%%%%%%%%%%%%%%%%%%%%%%%%

% Los logos tienen posiciones relativas al nombre de la escuela.
% Cada imagen esta desplazada con respecto al texto, en este caso nombre de la universidad.
% No se necesitan paquetes adicionales, el entorno estándar para imágenes de LaTeX puede hacerlo.
% El truco esta en definir una imagen de tamaño cero, asi no afecta al centrar los títulos.
\def\logoUNAM{%
  \begin{picture}(0,0)\unitlength=1cm
    \put (-3.5,-3) {\includegraphics[width=8em]{images/escudo-unam}}
  \end{picture}
}

\def\logoFI{%
  \begin{picture}(0,0)\unitlength=1cm
    \put (0.5,-3) {\includegraphics[width=8em]{images/escudo-fi}}
  \end{picture}
}

%%%%%%%%%%%%%%%%%%%%%%%%%%%%%%%%%%%%%%%%%%%%%%%%%%%%%%%%%%%%%%%%%%%%%%%%%%%%%%%%%%%%%%%%%

\author{LIDSOL}                                             % Autor del curso.
\title{Linux básico}         % Titulo del curso.
\date{20/08/24}                                             % Fecha de entrega.

\def\universidad{Universidad Nacional Autónoma de México}   % Nombre de la universidad.
\def\facultad{Facultad de Ingeniería}                       % Nombre de la facultad.
\def\semestre{2024-2}                                       % Semestre lectivo.
\def\laboratorio{Laboratorio de Investigación y Desarrollo del Software Libre}               % Nombre del laboratorio.
\makeatletter

%%%%%%%%%%%%%%%%%%%%%%%%%%%%%%%%%%%%%%%%%%%%%%%%%%%%%%%%%%%%%%%%%%%%%%%%%%%%%%%%%%%%%%%%%

\begin{document}
  % Titulo del documento con logos.
  \begin{center}
    \logoUNAM {\Large \universidad} \logoFI\par
    {\large \facultad}\par

    \laboratorio\par
    \semestre\par
    \@author\par
    \@date\par
    \@title
  \end{center}

  \hrulefill\par

  \pagenumbering{gobble}                              % Oculta el numero de pagina.
  \tableofcontents                                    % Crea el indice o tabla de contenido.

%%%%%%%%%%%%%%%%%%%%%%%%%%%%%%%%%%%%%%%%%%%%%%%%%%%%%%%%%%%%%%%%%%%%%%%%%%%%%%%%%%%%%%%%%

  \newpage
  \pagenumbering{arabic} % Muestra el numero de pagina.

  \section{Temario}

    \begin{enumerate}
        \item Primera clase
            \begin{enumerate}
                \item Introducción al Software Libre e historia del sistema
                    Linux. [30 min]
                \item Instalación de un ambiente Linux [1:30 h]
            \end{enumerate}
        \item Segunda clase
            \begin{enumerate}
                \item Sistema de Archivos [1 h]
                \item Arquitectura de un sistema tipo UNIX [30 min]
                \item Shell [30 min]
            \end{enumerate}
        \item Tercera clase
            \begin{enumerate}
                \item Shell (cont.) [1:30 h]
                \item Gestión de paquetes (nativos y Flatpak) [30 min]
            \end{enumerate}
        \item Cuarta clase
            \begin{enumerate}
                \item Edición de archivos [1 h]
                \item Gestión de servicios (SystemD) [1 h]
            \end{enumerate}
    \end{enumerate}

  \section{Objetivos}

    El curso está dirigido a personas sin conocimientos previos sobre el uso de sistemas Linux o tipo UNIX. Este tipo de personas tienen la necesidad o interés de utilizar este sistema en su día a día.

    Por ello, el objetivo es que los alumnos sean capaces de interactuar con un sistema Linux para uso personal.

  \section{Requisitos}

    \begin{enumerate}
        \item Una computadora personal propia, con cargador.
        \item 120 GB de espacio libre en su disco duro.
        \item Un respaldo de TODOS los archivos importantes de su computadora.
        \item De ser posible, una intalación ya realizada de un sistema Linux.
    \end{enumerate}

  \section{Metodología de enseñanza}

    Las sesiones estarán basadas en una serie de presentaciones teóricas y prácticas de manera que los estudiantes puedan afianzar sus conocimientos mientras reciben la información. Además, se contará con actividades poesteriores a la presentación para reforzar lo visto.

  \section{Evaluación}

    El curso se evaluará por medio de asistencia (75\%).

  \section{Recursos}

    \begin{thebibliography}{99}

      \bibitem{wolf}
      G. Wolf,
      \textit{Fundamentos de sistemas operativos}.

      \bibitem{ward}
      B. Ward,
      \textit{How Linux Works}.

      \bibitem{shotts}
      W. Shotts,
      \textit{The Linux command line: a complete introduction}.

      \bibitem{otw}
      OccupyTheWeb,
      \textit{Linux Basics for Hackers}.

    \end{thebibliography}
  \section{Calendario}

    Proporciona un calendario detallado con fechas importantes, como fechas límite de entrega de trabajos, fechas de exámenes, fin del curso, etc.

    \begin{center}
      \resizebox{\textwidth}{!}{
      \begin{tabular}{|c|c|c|c|c|c|}
      \hline
      & \textbf{Lunes 9 de septiembre} & \textbf{Martes 10 de septiembre} & \textbf{Miércoles 11 de septiembre} & \textbf{Jueves 12 de septiembre} & \textbf{Viernes 13 de septiembre} \\
      \hline
      12:00 - 13:00 & & & & & \\
      \hline
      13:00 - 14:00 & \cellcolor{yellow!25}Curso & \cellcolor{yellow!25}Curso & \cellcolor{yellow!25}Curso & \cellcolor{yellow!25}Curso & \cellcolor{yellow!25}Curso \\
      \hline
      14:00 - 15:00 & \cellcolor{yellow!25}Curso & \cellcolor{yellow!25}Curso & \cellcolor{yellow!25}Curso & \cellcolor{yellow!25}Curso & \cellcolor{yellow!25}Curso \\
      \hline
      15:00 - 16:00 & & & & & \\
      \hline
      \end{tabular}
      }
    \end{center}

    \begin{center}
      \resizebox{\textwidth}{!}{
      \begin{tabular}{|c|c|c|c|c|c|}
      \hline
      & \textbf{Lunes 16 de septiembre} & \textbf{Martes 17 de septiembre} & \textbf{Miércoles 18 de septiembre} & \textbf{Jueves 19 de septiembre} & \textbf{Viernes 20 de septiembre} \\
      \hline
      12:00 - 13:00 & & & & & \\
      \hline
      13:00 - 14:00 & \cellcolor{yellow!25}Curso & \cellcolor{yellow!25}Curso & \cellcolor{yellow!25}Curso & \cellcolor{yellow!25}Curso & \cellcolor{yellow!25}Curso \\
      \hline
      14:00 - 15:00 & \cellcolor{yellow!25}Curso & \cellcolor{yellow!25}Curso & \cellcolor{yellow!25}Curso & \cellcolor{yellow!25}Curso & \cellcolor{yellow!25}Curso \\
      \hline
      15:00 - 16:00 & & & & & \\
      \hline
      \end{tabular}
      }
    \end{center}

  % \section{Retroalimentación}
  %
  %   Incluye una sección para recopilar la retroalimentación de los estudiantes sobre el curso. Esto puede ayudarte a mejorar futuras ediciones del curso y a ajustar el contenido y la metodología de enseñanza según las necesidades y preferencias de los estudiantes.

%%%%%%%%%%%%%%%%%%%%%%%%%%%%%%%%%%%%%%%%%%%%%%%%%%%%%%%%%%%%%%%%%%%%%%%%%%%%%%%%%%%%%%%%%
\end{document}

%%%%%%%%%%%%%%%%%%%%%%%%%%%%%%%%%%%%%%%%%%%%%%%%%%%%%%%%%%%%%%%%%%%%%%%%%%%%%%%%%%%%%%%%%
% Autor:        Aguilar Enriquez, Paul Sebastian a.k.a. Penserbjorne
% Autor:        @yens7
% Autor:        Calderón Olalde, Enrique Job @ksobrenat32
% Fecha:        05/01/2024
% Descripción:  Plantilla base para cursos.
%%%%%%%%%%%%%%%%%%%%%%%%%%%%%%%%%%%%%%%%%%%%%%%%%%%%%%%%%%%%%%%%%%%%%%%%%%%%%%%%%%%%%%%%%

\documentclass[a4paper,11pt]{article}                 % Papel tamaño carta, texto de 11pt.

\usepackage[top=2cm, bottom=2cm, left=2.2cm, right=2.2cm]{geometry} % Margenes
\usepackage[T1]{fontenc}                              % Indicamos la codificación de las fuentes.
\usepackage[utf8x]{inputenc}                          % Definimos la codificación.
\usepackage{lmodern}                                  % Para poder usar acentos.
\usepackage[spanish]{babel}                           % Usaremos idioma español.
\usepackage{amsmath}                                  % Para formulas matemáticas.
\usepackage{graphicx}                                 % Para imágenes.
\usepackage{float}                                    % Para posicionar objetos.
\usepackage{booktabs}                                 % Para formatear tablas.
\usepackage{hyperref}                                 % Para enlaces y referencias.
\usepackage[table]{xcolor}


%%%%%%%%%%%%%%%%%%%%%%%%%%%%%%%%%%%%%%%%%%%%%%%%%%%%%%%%%%%%%%%%%%%%%%%%%%%%%%%%%%%%%%%%%

% Los logos tienen posiciones relativas al nombre de la escuela.
% Cada imagen esta desplazada con respecto al texto, en este caso nombre de la universidad.
% No se necesitan paquetes adicionales, el entorno estándar para imágenes de LaTeX puede hacerlo.
% El truco esta en definir una imagen de tamaño cero, asi no afecta al centrar los títulos.
\def\logoUNAM{%
  \begin{picture}(0,0)\unitlength=1cm
    \put (-3.5,-3) {\includegraphics[width=8em]{images/escudo-unam}}
  \end{picture}
}

\def\logoFI{%
  \begin{picture}(0,0)\unitlength=1cm
    \put (0.5,-3) {\includegraphics[width=8em]{images/escudo-fi}}
  \end{picture}
}

%%%%%%%%%%%%%%%%%%%%%%%%%%%%%%%%%%%%%%%%%%%%%%%%%%%%%%%%%%%%%%%%%%%%%%%%%%%%%%%%%%%%%%%%%

\author{LIDSoL}                                             % Autor del curso.
\title{Sysadmin GNU/Linux básico}         % Titulo del curso.
%\date{01/01/00}                                             % Fecha de entrega.

\def\universidad{Universidad Nacional Autónoma de México}   % Nombre de la universidad.
\def\facultad{Facultad de Ingeniería}                       % Nombre de la facultad.
\def\semestre{2024-2}                                       % Semestre lectivo.
\def\laboratorio{Laboratorio de Investigación y Desarrollo del Software Libre}               % Nombre del laboratorio.
\makeatletter

%%%%%%%%%%%%%%%%%%%%%%%%%%%%%%%%%%%%%%%%%%%%%%%%%%%%%%%%%%%%%%%%%%%%%%%%%%%%%%%%%%%%%%%%%

\begin{document}
  % Titulo del documento con logos.
  \begin{center}
    \logoUNAM {\Large \universidad} \logoFI\par
    {\large \facultad}\par

    \laboratorio\par
    \semestre\par
    \@author\par
    \@date\par
    \@title
  \end{center}

  \hrulefill\par

  \pagenumbering{gobble}                              % Oculta el numero de pagina.
  \tableofcontents                                    % Crea el indice o tabla de contenido.

%%%%%%%%%%%%%%%%%%%%%%%%%%%%%%%%%%%%%%%%%%%%%%%%%%%%%%%%%%%%%%%%%%%%%%%%%%%%%%%%%%%%%%%%%

  \newpage
  \pagenumbering{arabic} % Muestra el numero de pagina.

  \section{Temario}
    \begin{enumerate}
      \item Linea de comandos [7.5 h]
        \begin{enumerate}
          \item Estructura de directorios en GNU/Linux [0.5 h]
          \item Usuarios y permisos [1 h]
          \item Comandos básicos [2 h]
          \item Editor de texto [1 h]
          \item Instalación de programas [1 h]
          \item Shell [2 h]
        \end{enumerate}
      \item Sistema Debian GNU/Linux [12.5 h]
        \begin{enumerate}
          \item Systemd [1 h]
          \item Administración de servicios [2.5 h]
          \item Redes y Firewall [3 h]
          \item Almacenamiento y sistemas de archivos [2 h]
          \item Métodos de respaldo de información [2 h]
          \item Contenedores [2 h]
        \end{enumerate}
    \end{enumerate}

  \section{Objetivos}
    El curso esta dirigido hacia personas con conocimientos de GNU/Linux quienes se deseen adentrar a la administración y mantenimiento de servidores.

    Los alumnos obtendrán conocimientos sobre la administración de sistemas GNU/Linux para su uso como servidor.

  \section{Requisitos}
    \begin{itemize}
      \item Conocimientos básicos de GNU/Linux.
      \item Computadora con máquina virtual con Debian 12 (Se enviarán instrucciones).
    \end{itemize}

  \section{Metodología de enseñanza}
    Se utiliza un sistema teórico-práctico en el cual inicialmente se imparte la parte teórica para posteriormente asignar ejercicios de reforzamiento.

    Se finaliza con un proyecto para reforzar la totalidad de conocimientos adquiridos en el curso.

  \section{Evaluación}
    El curso se evalúa por medio de la asistencia (mínima del 80\%) al mismo y entrega del proyecto, siendo ambos requisitos para la obtención de constancia.

  \section{Recursos}
    \begin{itemize}
      \item Libro 1
    \end{itemize}

  \section{Calendario}
    \begin{center}
      \resizebox{\textwidth}{!}{
      \begin{tabular}{|c|c|c|c|c|c|}
      \hline
      & \textbf{Lunes 29 de Julio} & \textbf{Martes 30 de Julio} & \textbf{Miércoles 31 de Julio} & \textbf{Jueves 1 de Agosto} & \textbf{Viernes 2 de Agosto} \\
      \hline
      09:00 - 10:00 & & & & & \\
      \hline
      10:00 - 11:00 & \cellcolor{yellow!25}Curso & \cellcolor{yellow!25}Curso & \cellcolor{yellow!25}Curso & \cellcolor{yellow!25}Curso & \cellcolor{yellow!25}Curso \\
      \hline
      11:00 - 12:00 & \cellcolor{yellow!25}Curso & \cellcolor{yellow!25}Curso & \cellcolor{yellow!25}Curso & \cellcolor{yellow!25}Curso & \cellcolor{yellow!25}Curso \\
      \hline
      12:00 - 13:00 & \cellcolor{yellow!25}Curso & \cellcolor{yellow!25}Curso & \cellcolor{yellow!25}Curso & \cellcolor{yellow!25}Curso & \cellcolor{yellow!25}Curso \\
      \hline
      13:00 - 14:00 & \cellcolor{yellow!25}Curso & \cellcolor{yellow!25}Curso & \cellcolor{yellow!25}Curso & \cellcolor{yellow!25}Curso & \cellcolor{yellow!25}Curso \\
      \hline
      14:00 - 15:00 & & & & & \\
      \hline
      \end{tabular}
      }
    \end{center}

  %\section{Retroalimentación}
    %Incluye una sección para recopilar la retroalimentación de los estudiantes sobre el curso. Esto puede ayudarte a mejorar futuras ediciones del curso y a ajustar el contenido y la metodología de enseñanza según las necesidades y preferencias de los estudiantes.

%%%%%%%%%%%%%%%%%%%%%%%%%%%%%%%%%%%%%%%%%%%%%%%%%%%%%%%%%%%%%%%%%%%%%%%%%%%%%%%%%%%%%%%%%
\end{document}